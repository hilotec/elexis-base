%% LyX 1.6.5 created this file.  For more info, see http://www.lyx.org/.
%% Do not edit unless you really know what you are doing.
\documentclass[ngerman]{article}
\usepackage[T1]{fontenc}
\usepackage[latin9]{inputenc}
\usepackage{listings}
\usepackage[letterpaper]{geometry}
\geometry{verbose,tmargin=1.5cm,bmargin=1.5cm,lmargin=1cm,rmargin=1cm}
\usepackage{amsthm}
\usepackage{amsmath}

\makeatletter
%%%%%%%%%%%%%%%%%%%%%%%%%%%%%% Textclass specific LaTeX commands.
\numberwithin{equation}{section}
\numberwithin{figure}{section}

\makeatother

\usepackage{babel}

\begin{document}
\selectlanguage{ngerman}

\section*{Aufsetzen von OpenVPN auf einem Redhat-Server}


\section{Voraussetzungen}

Benutzer-Namen und Passwort f�r RedHat-Server sind bekannt. Benutzer
muss z.B. via sudo Zugriff mit System-Rechten haben.


\section{Installation auf einem RedHat Server (5.4)}

Die meisten Informationen kommen von: http://www.packtpub.com/article/installing-openvpn-on-linux-and-unix-systems-1

Folgende Befehle als root ausf�hren:


\begin{lstlisting}
wget http://www.fedorafaq.org/samples/yum.conf
rpm -Uvh http://www.fedorafaq.org/yum
yum install openvpn
\end{lstlisting}



\section{Einfacher Test, ob es l�uft oder nicht}

S.a. http://www.openvpn.net/index.php/open-source/documentation/miscellaneous/78-static-key-mini-howto.html

Achtung: Damit kann auf dem Server jedoch nur 1 Client laufen!

Generate a static key:


\begin{lstlisting}
cd /etc/openvpn && openvpn --genkey --secret elexis_test_labo.key
\end{lstlisting}


Copy the static key to both client and server, over a pre-existing
secure channel. Server configuration file /etc/openvpn/elexis\_test\_labo.conf


\begin{lstlisting}
# Konfiguration einfacher OpenVPN Test-Server fuer elexis
# Server-Seite
dev tun
ifconfig 10.87.53.102   10.87.53.101 
secret elexis_test_labo.key
\end{lstlisting}


Danach kann man lokal testen, mit Hilfe von sftp 100.87.0.53.101

Client configuration file Pfad-Zu-Installation-Von-OpenVPN/etc/openvpn/elexis\_test\_arzt.ovpn
(ovpn ist f�r Windows-Clients, f�r Linux-Clients w�re es ebenfalls
{*}.conf)


\begin{lstlisting}
# Konfiguration einfacher OpenVPN Test-Server fuer Labor
# Client
dev tun
remote 62.202.1.152
#
#        Client (Arzt) ->       Server (Labo)
ifconfig 10.87.53.102   10.87.53.101
secret Labor_test_1.key
\end{lstlisting}


Firewall-Einstellungen wurde so ge�ndert, dass sie auf dem Port 1194
sowohl UDP als auch TCP in beide Richtung zulassen.

Falls auf dem Client alles gut l�uft, k�nnen die ge�ffneten Dienste
(FTP) auf dem Server 10.87.53.101 benutzt werden.


\subsection{Anpassungen f�r mehrere Clients}

Dazu m�ssen auf dem Server Schl�ssel generiert werden. Die Schl�ssel
f�r den Server m�ssen nur einmal generiert werden. Dazu dient folgendes
Script


\begin{lstlisting}[basicstyle={\ttfamily},showstringspaces=false]
#! /bin/bash 
# 2010-04-10 Niklaus Giger 
# Die Dateien /etc/openvpn/vars und /etc/openvpn/server.conf anpassen. 
# Dann diese Datei als root ausf�hren 
rm -rf /etc/openvpn/easy-rsa 
mkdir  /etc/openvpn/easy-rsa 
cp -rp /usr/src/redhat/BUILD/openvpn-2.1.1/easy-rsa/2.0/* /etc/openvpn/easy-rsa 
rm -rf /etc/openvpn/easy-rsa/vars 
cd /etc/openvpn/easy-rsa 
. /etc/openvpn/vars 
/etc/openvpn/easy-rsa/clean-all 
# Hier sollte bei ein paar Fragen nur noch okay geklickt werden m�ssen 
/etc/openvpn/easy-rsa/build-ca --batch 
# Den Server-Key erstellen
/etc/openvpn/easy-rsa/build-key-server  --batch server 
# Der n�chste Befehl kann auch auf einer schnellen Maschine mehrere Minuten dauern 
/etc/openvpn/easy-rsa/build-dh 
\end{lstlisting}


Folgendes Script muss jeweils pro neuen Clients ausgef�hrt werden.


\begin{lstlisting}[basicstyle={\ttfamily},breaklines=true,showstringspaces=false]
#! /bin/bash
# 2010-04-10 Niklaus Giger
#
# *.key-Dateien sollten geheim gehalten werden
# *.crt und *.pem-Dateien sind �ffentlich
# Am besten keine Leerzeichen beim Kundenamen haben.
#
KundenName="client3"

#------------------------------------------------------------------------------
# Hier noch ihre Parameter beim UserAdd hinzuf�gen
# Habe mal -m zum Erstellen des Home-Verzeichnisses genommen
#------------------------------------------------------------------------------
/usr/sbin/useradd --create-home  $KundenName
# passwd $KundenName ??

#------------------------------------------------------------------------------
# Ab hier sollte es nichts mehr zum Andern geben
#------------------------------------------------------------------------------

# Dann diese Datei als root ausf�hren
cd /etc/openvpn/easy-rsa
. /etc/openvpn/vars
#Jetzt noch den ersten Test-Clients erstellen
/etc/openvpn/easy-rsa/build-key --batch $KundenName

# What follows is a so called Here document (or variable) of bash
confFile=$(cat <<EndOfVar
##############################################
# Config-File for Labor OpenVPN          #
##############################################

client
dev tun

# Windows needs the TAP-Win32 adapter name
# from the Network Connections panel
# if you have more than one.  On XP SP2,
# you may need to disable the firewall
# for the TAP adapter.
;dev-node MyTap

proto udp

remote 62.202.1.152 1194

# Keep trying indefinitely to resolve the
# host name of the OpenVPN server.  Very useful
# on machines which are not permanently connected
# to the internet such as laptops.
resolv-retry infinite

nobind

# Try to preserve some state across restarts.
persist-key
persist-tun

ca   Labor-ca.crt
cert $KundenName.crt
key  $KundenName.key

ns-cert-type server
comp-lzo

# Set log file verbosity.
verb 3
 
# Silence repeating messages
mute 20
mute-replay-warnings
EndOfVar)

echo "$confFile" > /etc/openvpn/keys/$KundenName.ovpn
unix2dos /etc/openvpn/keys/$KundenName.ovpn
cp /etc/openvpn/keys/ca.crt /home/$KundenName/Labor-ca.crt
cp /etc/openvpn/keys/$KundenName.crt /home/$KundenName
cp /etc/openvpn/keys/$KundenName.key /home/$KundenName
chown $KundenName /home/$KundenName/$KundenName.*
echo "$confFile" > /home/$KundenName/$KundenName.ovpn
echo /home/$KundeName/$KundenName.ovpn
echo "Created config/key-files for $KundenName in /home/$KundenName"
ls -l /home/$KundenName
echo $clientConf
\end{lstlisting}


Zum Testen auf dem Client openvpn starten. Etwa folgendes sollte im
Log zu sehen sein:


\begin{lstlisting}[basicstyle={\small},breaklines=true,showstringspaces=false]
Sun Apr 11 12:01:41 2010 OpenVPN 2.1.1 i686-pc-mingw32 [SSL] [LZO2] [PKCS11] built on Dec 11 2009 
Sun Apr 11 12:01:41 2010 NOTE: OpenVPN 2.1 requires '--script-security 2' or higher to call user-defined scripts or executables 
Sun Apr 11 12:01:41 2010 LZO compression initialized Sun Apr 11 12:01:41 2010 Control Channel MTU parms [ L:1542 D:138 EF:38 EB:0 ET:0 EL:0 ] 
Sun Apr 11 12:01:41 2010 Data Channel MTU parms [ L:1542 D:1450 EF:42 EB:135 ET:0 EL:0 AF:3/1 ] 
Sun Apr 11 12:01:41 2010 Local Options hash (VER=V4): '41690919' 
Sun Apr 11 12:01:41 2010 Expected Remote Options hash (VER=V4): '530fdded' 
Sun Apr 11 12:01:41 2010 Socket Buffers: R=[8192->8192] S=[8192->8192] 
Sun Apr 11 12:01:41 2010 UDPv4 link local: [undef] 
Sun Apr 11 12:01:41 2010 UDPv4 link remote: 172.25.1.134:1194 
Sun Apr 11 12:01:41 2010 TLS: Initial packet from 172.25.1.134:1194, sid=74d0481d b9c0622a 
Sun Apr 11 12:01:42 2010 VERIFY OK: depth=1, /C=CH/ST=ZH/L=Zuerich/O=Labor/CN=Labor_CA/emailAddress=ihr.administrator@Labor.ch 
Sun Apr 11 12:01:42 2010 VERIFY OK: nsCertType=SERVER 
Sun Apr 11 12:01:42 2010 VERIFY OK: depth=0, /C=CH/ST=ZH/L=Zuerich/O=Labor/CN=server/emailAddress=ihr.administrator@Labor.ch 
Sun Apr 11 12:01:42 2010 Data Channel Encrypt: Cipher 'BF-CBC' initialized with 128 bit key 
Sun Apr 11 12:01:42 2010 Data Channel Encrypt: Using 160 bit message hash 'SHA1' for HMAC authentication 
Sun Apr 11 12:01:42 2010 Data Channel Decrypt: Cipher 'BF-CBC' initialized with 128 bit key 
Sun Apr 11 12:01:42 2010 Data Channel Decrypt: Using 160 bit message hash 'SHA1' for HMAC authentication 
Sun Apr 11 12:01:42 2010 Control Channel: TLSv1, cipher TLSv1/SSLv3 DHE-RSA-AES256-SHA, 1024 bit RSA 
Sun Apr 11 12:01:42 2010 [server] Peer Connection Initiated with 172.25.1.134:1194 
Sun Apr 11 12:01:44 2010 SENT CONTROL [server]: 'PUSH_REQUEST' (status=1) 
Sun Apr 11 12:01:44 2010 PUSH: Received control message: 'PUSH_REPLY,route 172.23.45.1,topology net30,ifconfig 172.23.45.14 172.23.45.13' 
Sun Apr 11 12:01:44 2010 OPTIONS IMPORT: --ifconfig/up options modified S
un Apr 11 12:01:44 2010 OPTIONS IMPORT: route options modified 
Sun Apr 11 12:01:44 2010 ROUTE default_gateway=192.168.102.1 
Sun Apr 11 12:01:44 2010 TAP-WIN32 device [LAN-Verbindung 3] opened: \\.\Global\{CD34DAE6-E684-4EC2-91A5-3FFAA13C25F3}.tap 
Sun Apr 11 12:01:44 2010 TAP-Win32 Driver Version 9.6 
Sun Apr 11 12:01:44 2010 TAP-Win32 MTU=1500 
Sun Apr 11 12:01:44 2010 Notified TAP-Win32 driver to set a DHCP IP/netmask of 172.23.45.14/255.255.255.252 on interface {CD34DAE6-E684-4EC2-91A5-3FFAA13C25F3} [DHCP-serv: 172.23.45.13, lease-time: 31536000] 
Sun Apr 11 12:01:44 2010 Successful ARP Flush on interface [17] {CD34DAE6-E684-4EC2-91A5-3FFAA13C25F3} 
Sun Apr 11 12:01:49 2010 TEST ROUTES: 1/1 succeeded len=1 ret=1 a=0 u/d=up 
Sun Apr 11 12:01:49 2010 C:\WINDOWS\system32\route.exe ADD 172.23.45.1 MASK 255.255.255.255 172.23.45.13 
Sun Apr 11 12:01:49 2010 ROUTE: CreateIpForwardEntry succeeded with dwForwardMetric1=30 and dwForwardType=4 
Sun Apr 11 12:01:49 2010 Route addition via IPAPI succeeded [adaptive] 
Sun Apr 11 12:01:49 2010 Initialization Sequence Completed 
\end{lstlisting}



\subsection{Testumgebung}

RedHat 30-Tag-Lizenz, RedHat 5.4-DVD (2.6 GB) runtergeladen und auf
MacBookPro via KVM/virt-manager installiert.
\begin{itemize}
\item 1855755 30-day Unsupported Evaluation Red Hat Enterprise Linux (Up
to 2 Sockets) 
\item 03/29/2010 04/27/2010 Info zur Verl�ngerung 
\item RHN 363fcca9327b83cc
\end{itemize}


\end{document}
