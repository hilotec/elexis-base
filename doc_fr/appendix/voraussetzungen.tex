% *******************************************************************************
% * Copyright (c) 2007 by Elexis
% * All rights reserved. This document and the accompanying materials
% * are made available under the terms of the Eclipse Public License v1.0
% * which accompanies this distribution, and is available at
% * http://www.eclipse.org/legal/epl-v10.html
% *
% *  $Id: voraussetzungen.tex 3094 2007-09-04 10:03:03Z rgw_ch $
%
%*******************************************************************************
% !Mode:: "TeX:UTF-8" (encoding info for WinEdt)

\section{Prérequis minimale au hardware}
\label{systemvoraussetzungen}
\begin{itemize}
 \item Un PC à peu près actuel (cadence min. de 1GHz, 512 MB RAM min , 1GB recommandé) disque dur de 1GB min.
\item  Carte graphique 1024 x 768 pixel min (1280 x 1024 recommandé) et un écran convenable (par ex. 17 pouces TFT ou plus).
\item Recommandé : Imprimante avec alimentation bac 1 pour du papier A5 (Ordonnances, Certificats) et bac 2 et eventuellement bac 3 pour du papier A4 (Lettres, Factures)
\item Recommandé : Imprimante d'étiquettes.
\item Recommandé : Drive externe pour sauvegarde .
\item Recommandé : Accès Internet protégé par une \textbf{Hardware}-Firewall (une protection par simple Personal Firewall est  \textbf{explicitement déconseillée}) (voir p. \pageref{sicherheit}).
\end{itemize}

\section{Systèmes d'exploitation supportés}
Elexis fonctionne en principe dans tout les systèmes d'exploitation pour lesquels une Environment Version 1.5 ou plus avancée du Java Runtime existe . En particulier ce sont les systèmes suivants :
\begin{itemize}
\item  Windows 2000, XP, Vista
\item  Macintosh OS à partir de 10.4 (Tiger)  \footnote{Sous MacOS-X une intégration OpenOffice n'est malheureusement pas possible.}
\item Linux (SuSE à partir de 9.3 ou Xubuntu/Kubuntu à partir de 6.06)\footnote{Attention : Sous Linux avec Gnome-Desktop (comme Ubuntu) l'intégration OpenOffice ne fonctionne pas, par contre sous KDE (Kubuntu) et Xfce(Xubuntu) ça fonctionne. }
\end{itemize}
Dans ces systèmes d'exploitation Elexis peut être installé directement avec une des versions complètes mises à disposition. L'installation sur d'autres systèmes nécessitera plus ou moins de travail 'manuel'. Veuillez prendre en considération que nos offres forfaitaires et contrats de maintenances ne couvrent que les systèmes qui remplissent les conditions mentionnés ci-dessus en ce qui concerne la Hardware et le système d'exploitation. 