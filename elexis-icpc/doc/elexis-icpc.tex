% *******************************************************************************
% * Copyright (c) 2007 by Elexis
% * All rights reserved. This document and the accompanying materials
% * are made available under the terms of the Eclipse Public License v1.0
% * which accompanies this distribution, and is available at
% * http://www.eclipse.org/legal/epl-v10.html
% *
% * Contributors:
% *    G. Weirich
% *
% *  $Id: elexis-icpc.tex 4360 2008-09-02 17:16:05Z rgw_ch $
% *******************************************************************************
% !Mode:: "TeX:UTF-8" (encoding info for WinEdt)

\documentclass[a4paper]{scrartcl}
\usepackage{german}
\usepackage[utf8]{inputenc}
\usepackage{makeidx}
\makeindex
% Hier ein etwas skurriler Block, der dazu dient, die Unterschiede
% zwischen pdflatex und latex auszubügeln
% Grafiken müssen als png oder gif (für pdflatex) und als eps (für Latex)
% vorhanden sein. Die Endung kann man beim \includegraphics jeweils weglassen,
% das System nimmt je nach Renderer die geeignete Variante.

\newif\ifpdf
\ifx\pdfoutput\undefined
	\pdffalse              	%normales LaTeX wird ausgeführt
\else
	\pdfoutput=1
	\pdftrue               	%pdfLaTeX wird ausgeführt
\fi

\ifpdf
	\usepackage[pdftex]{graphicx}
	\DeclareGraphicsExtensions{.pdf,.jpg,.png}
\else
	\usepackage[dvips]{graphicx}
	\DeclareGraphicsExtensions{.eps}
\fi

\usepackage{floatflt}
\usepackage{wrapfig}
\usepackage[]{hyperref}
\usepackage{color}
\begin{document}
\title{ICPC-2 in Elexis}
\author{Gerry Weirich}
\maketitle

\section{Einführung}
ICPC-2, der \textbf{I}nternational \textbf{C}lassification of \textbf{P}rimary \textbf{C}are ist ein Codierungswerkzeug für die Bedürfnise der Grundversorger. Der Ansatz des Hausarztes ist meist 'Grundzentriert' und nicht 'Diagnosezentriert': Die erste Frage ist 'Warum kommt der Patient'. Dementsprechend kommen hier oft Bezeichnungen zur Anwendung, die in bekannten Codesystemen wie ICD-10 oder CHOP nicht gut codiert werden können ('Unwohlsein', oder 'Angst vor Krebs' etc.). ICPC erlaubt also die Abbildung der Tätigkeit des Hausarztes und ist darum ein geeignetes Werkzeug, um sowohl Abrechnungsstatistische als auch wissenschaftliche Daten dieser Tätigkeit zu erheben. Gleichzeitig ist ICPC nach einer gewissen Eingewöhnungszeit auch sehr einfach in der Anwendung - Die Codierung ist viel weniger Zeitraubend, als mit ICD-10, da der Code weniger umfangreich und stärker 'Grundzentriert' ist, wie wir unten weiter ausführen werden.

\medskip
ICPC ist lizenzpflichtig. Inhaber der Lizenz für die Schweiz ist die Schweizerische Gesellschaft für Allgemeinmedizin (SGAM). Man kann sich dort gegen eine Gebühr registrieren lassen und erhält dann eine Datenbank mit der aktuellen Codeversion. Näheres s. http://www.icpc.ch.

\section{Codierungsprinzipien}
Basis der Codierung ist eine 'Episode'. Diese ist eine alte Bekannte, auch wenn Sie sie vielleicht eher als 'Problem' in einer Problemliste aufgeführt haben. Die Elexis-Implementation von ICPC verwendet deshalb auch gleich eine Problemliste anstatt einer 'Episodenliste' -- beides ist aber, his auf die Bezeichnung -- dasselbe.

\medskip

Wenn ein Patient wegen eines Problems behandelt wird, dann ist dies ein 'Encounter'. Mangels eines besseren deutschen Worts bezeichnen wir dies hier weiterhin als Encounter. Ein Encounter ist nicht dasselbe wie eine Konsultation: Während einer Konsutlation können mehrere Probleme behandelt werden, was mehreren Encounters entspricht. Ein Problem dauert also eine bestimmte Zeit im Leben eines Menschen, und ein Encounter ist jeder Punkt innerhalb dieses Zeitraums, an dem er den Arzt wegen dieses Problems aufsucht.

\medskip

Für jeden Encounter innerhalb einer Konsultation werden nun nach ICPC drei Elemente codiert:
\begin{itemize}
\item \textbf{RFE}: Reason for encounter. Warum ist der Patient gekommen? Hier geht es also nicht darum, eine medizinische Diagnose anzugeben, sondern vielmehr einen ICPC2-Code, der möglichst genau das abbildet, was uns der Patient gesagt hat.
\item \textbf{Diagnose}: Dies enthält nun die Interpretation des Arztes in dessen Denkschema: Was denkt der Arzt, was hinter den Beschwerden des Patienten steckt.
\item \textbf{Procedere}: Dies ist der vorläufige Behandlungsplan: Was wurde getan oder geplant?
\end{itemize}

Alle drei Elemente können für ein- und dasselbe Problem bei jedem Encounter wieder anders sein. unter http://www.icpc.ch/index.php?id=20 finden Sie eine Grafik, die diese Zusammenhänge sehr schön zeigt.

\section{Anwendung in Elexis}
Nach Installation des Plugins finden Sie eine ICPC-Perspektive vor. Diese ist nur als Beispiel zu verstehen, und kann wie in Elexis üblich, beliebig an Ihre Vorlieben und Ihren Bildschirm angepasst werden. Sie können Views entfernen oder zufügen, oder Views anders anordnen.
Zur Anwendung von ICPC-2 benötigen Sie aber mindestens die Views 'Konsultation', 'Probleme', 'Encounters' und 'ICPC2-Codes'.

\subsection{Ein Problem erstellen und bearbeiten}
Klicken Sie in der Probleme-View auf den roten Stern. Sie können dann ein neues Problem erfassen. Für jedes Problem kann ein Name, ein Beginn-Datum und eine Nummer angegeben werden. Probleme werden in der Probleme-View immer anhand der Nummer sortiert. Es sind auch hierarchische Nummern (1.1.2 oder so) möglich. Das Beginn-Datum kann als genaues Datum oder auch nur als Jahr angegeben werden. Standardmässig wird das aktuelle Datum genommen.

\medskip

Wenn Sie mit der rechten Maustaste auf ein Problem klicken


\subsection{Ein Problem in einer Konsultation behandeln}
Ziehen Sie ein Problem aus der Problemliste mit der Maus ins Konsultationsfenster der aktuellen Konsultation. Es erscheint der Text Problem: gefolgt vom grün unterlegten Namen des Problems.

\subsection{Ein Encounter codieren}
b

\subsection{Einem Problem eine Krankenkassendiagnose zuordnen}
Üblicherweise wollen (und eigentlich müssen!) wir den Krankenkassen nur eine stark vereinfachte Version der Diagnose liefern. In der Schweiz hat sich der Tessiner Code für die auf den Rechnungen erscheinende Diagnose eingebürgert, im UVG-Bereich auch der ICD-10. Sie können einen solchen Code aus dem Diagnosen-Fenster auf ein Problem ziehen, dann wird künftig immer wenn dieses Problem behandelt wird, der so zugeordete Diagnosecode automatisch ins 'Diagnose'-Feld der Konsultation (und damit auf die Rechnung) gesetzt.

\subsection{Ein Problem durch die Zeit verfolgen}
Klicken Sie auf den Filter-Knopf in der Probleme-View und markieren Sie das interessierende Problem.  In der Konsultationen-View werden dann nur noch diejenigen Konsultationen aufgelistet, in denen dieses Problem behandelt wurde.

\subsection{Die Entwicklung eines Problems chronologisch aufzeigen}
In der 'Encounters'-View wird für das aktuell markierte Problem die entsprechende RFE, Diagnose und Procedere chronologisch aufgelistet gezeigt.


\end{document} 