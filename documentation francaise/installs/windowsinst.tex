% !Mode:: "TeX:UTF-8" (encoding info for WinEdt)
\section{PostgreSQL sous Windows}
de point en point: Installer PostgreSQL pour Elexis sur Windows.


\textbf{préparer l'installation psql}

\begin{itemize}
\item téléchargez postgresql-8.1.5-1.zip
\item déballer et executer le postgresql-8.1. msi qui s'y trouve.
\item installer selon d:{/}test (le disque dur doit être formaté en NTFS), installer toutes les options comme proposées
\item  {[x]} installer comme services
\item  nom de l'utilisateur postgres
\item  mot de passe aussi postgres (devait naturellement être un autre si vous travallez dans l'environement original)
\item  Remplacer le mot de passe? {[}Non{]}
\item  {[x]} initialisez le cluster da la base des données
\item  {[x]} accepter toutes les connexions sur toutes les adresses IP
\item  Locale German, Switzerland
\item  Encoding: UTF-8
\item  mot de passe superuser: elexisadmin
\item  {[x]} pl/pgsql
\item  ne choisissez pas de contrib-module
\item  {[x]} Admin81
\end{itemize}

\textbf{l'installation se déroule}

Sous d:/test/data configurer le fichier pg\_hba.conf en
ajoutant  la ligne suivante sous IVP4 local connections :


%ajouter la ligne suivante sous IPV4 local connections :
host all all 192.168.0.1/24 md5


(Ceci part de l'idée que le réseau local se trouve dans 192.168.0.x et que vous donnez accès à ce serveur à tous les ordinateurs se trouvant dans ce réseau.) 
S'il existe une personal firewall : ouvrir le port 5432



\textbf{installer la base de données:}

\begin{itemize}
\item démarrer pgadmin3
\item double clic gauche sur le localhost du serveur de la PostgreSQL Database
\item mot de passe: elexisadmin
\item Clique droite sur \textit{Login roles}
\textit{new Login role}
\item Role Name: elexisuser
\item Password: meinepraxis
\item {[x]} can create database objects
\item clic droit \textit{Databases}
\textit{new database}
\item nom: elexis
\item Owner: elexisuser
\item Encoding: UTF-8
\item Tablespace: pg\_default
\item démarrer elexis
\item menu: fichier - connexion 
\item type: postgreSQL
\item adresse du serveur : localhost
\item nome de la base de données : elexis [suivant]
\item nom d'ultilisateur: elexisuser
\item mot de passe: meinepraxis
\item [terminer]
\end{itemize}

S'il y a quelques messages d'erreur, fermez-les. Le programme se ferme ensuite. Redémarrer Elexis. 

Argh, maintenant il y a un bug, la nouvelle liaison n'a pas encore été sauvegardée :

\begin{itemize}
 \item Encore une fois le menu fichier- connexion
\item type: postgreSQL
\item adresse du serveur : localhost
\item nom de la base de données: elexis [suivant]
\item nom d'utilisateur: elexisuser
\item mot de passe : meinepraxis
\item [terminer]
\item fermer Elexis
\item redémarrer Elexis.
\end{itemize}
Maintenant la connexion avec PostgreSQL a été établie et la base de données est constituée. (Testez avec Fichier-Connexion où cette nouvelle connexion sera affichée.
-{>} Description comme ici : \href{http://www.elexis.ch/jp/content/view/51/47/}{ici}
(c'est un peu vieux mais toujours plus ou moins juste)

\chapter{Clients}
\section{Windows 2000/XP}
Sous Windows 2000 et XP home ou XP pro l'installation est relativement facile. Décidez-vous d'abord si vous avez besoin d'une version avec ou sans JRE (JAVA) et démarrez ensuite l'installateur avec un double-clic.
Pour la réponse à la question de quelle version vous avez besoin, veuillez prodécer de façon suivante :
Ouvrez un poste de travail (démarrer - exécuter -> cmd.exe)et introduisez :
java -version
Si vous recevez un message d'erreur ou un version < 1.5, vous nécessitez la version de Elexis qui contient le JRE.
En cas de doute on vous suggère de toujours utiliser la version avec JRE car ceci ne change rien dans votre ordinateur à par  de que ça demande un peu de place. Le JRE ne s'installera que dans le fichier local de Elexis et sera effacé avec celui-là lors de la désinstallation de Elexis. 
Après un double-clic sur l'installateur (elexis-win32-<version>-<jre>.exe) vous pouvez laisser en cas de doute toutes les options dans l'état 'par défaut'.
\subsection{Si quelque chose tourne mal}
Si jamais l'installation ne réussit pas du premier coup, il vaut mieux de recommencer à zéro. Donc, effacez complètement la base des données et réinstallez la. Démarrez Elexis lors du premier démarrage depuis la ligne de commande avec l'option suivante :
\begin{itemize}
\item elexis -clean\_all
\end{itemize}


Ceci permet d'effacer tous les restes d'une installation échouée. 
\subsection{Si ceci ne march pas}
\subsubsection{Support par E-Mail}

Le support pour l'installation par E-Mail reste gratuite jusqu'à ce que la phase béta est terminée.
\subsubsection{Support par télémaintenance}

Le support par télémaintenance (VNC) reste gratuit jusqu'à ce que la phase béta est terminée, mais vous devez installer vous-même le serveur VNC qui est nécessaire pour celà. Dans beaucoup de cas toute l'installation peut se faire à travers VNC.
Installation sur place :
Nous installons Elexis y inclus une base de données pour vous à un prix forfaitaire de CHF 250.-- pour les systèmes suivants. (S'y ajoutent les frais de voyage.) :
\begin{itemize}
 \item Windows 2000,
\item XP Home ou XP Pro
\item avec base de données hsql, mysql ou postgresql
\end{itemize}

\subsubsection{Windows Vista}
Nous ne pouvons pour le moment pas garantir que Elexis fonctionne sur Windows Vista. Pour cette raison nous ne conseillons actuellement pas encore ce système pour Elexis.

