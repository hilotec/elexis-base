% *******************************************************************************
% * Copyright (c) 2007 by Elexis
% * All rights reserved. This document and the accompanying materials
% * are made available under the terms of the Eclipse Public License v1.0
% * which accompanies this distribution, and is available at
% * http://www.eclipse.org/legal/epl-v10.html
% *
% *  $Id: extplugins.tex 2472 2007-06-03 09:48:14Z rgw_ch $
% *******************************************************************************
% !Mode:: "TeX:UTF-8" (encoding info for WinEdt)

Dieser Anhang listet einige bekannte Zusatzplugins für Elexis auf. Es muss nochmals betont werden, dass niemals alle existierenden Plugins bekannt sein können, und dass es darum auch keine abschliessende Liste geben kann. Wir möchten aber unabhängige Plugin-Entwickler ermutigen, ihre Plugins zu melden, damit wir sie in späteren Auflagen dieses Handbuchs mit aufführen können.
Wenden Sie sich bitte für weitere Informationen zu diesen Plugins an die bei jedem Plugin genannte Kontaktadresse. Bezugs- und Nutzungsbedingungen für externe Plugins können anderes sein, als für das Hauptprogramm (Beispielsweise können externe Plugins kostenpflichtig sein, und externe Plugins müssen auch nicht unbedingt quelloffen sein).

\section{ICPC-2}
Dieses Plugin implementiert den International Code for Primary Care in Elexis. ICPC-2 ist ein modernes, Praxisnahes Klassifikationsmodell für Konsultationsgründe, Diagnosen und Massnahmen. S, z.B. folgenden Link:
\href{http://www.primary-care.ch/pdf/2005/2005-10/2005-10-656.PDF}{http://www.primary-care.ch/pdf/2005/2005-10/2005-10-656.PDF}


ICPC-2 ist speziell auch für Statistik und hausärztliche Forschung geeignet. Allerdings ist ICPC-2 leider nicht frei verfügbar, sondern mit Lizenzgebühren belastet. Daher enthält dieses Plugin auch nicht den Code, sondern nur die Umgebung, in die der Code eingelesen und angewendet werden kann. Für die Lizensierung in der Schweiz ist die Schweizerische Gesellschaft für Allgemeinmedizin zuständig (\href{http://www.sgam.ch}{www.sgam.ch})

\section{Elexis Laborimport Risch}

Import von Laborwerten (und Parametern) des \href{http://www.risch.ch}{Labors Risch}.

\section{Elexis Laborimport Krech}

Import von Laborresultaten(und Parametern) aus den \href{http://www.labor.ch}{Labors Krech} , Kreuzlingen und MicroGen (und \textit{kompatible}).

\section{Elexis-Importer-PraxisDesktop}

Import von Patientendaten aus dem Programm \href{http://www.praxisdesktop.ch}{PraxisDesktop}. Dieses Plugin wird separat vertrieben.


 \section{Elexis-Importer-Vitomed}

 Import von Stammdaten des Programms Vitomed (\href{http://www.vitomed.ch}{www.vitomed.ch}


\section{Buchhaltung}
Einfache Praxisbuchhaltung. Dieses Plugin wird separat von Elexis vertrieben und ist kostenpflichtig. 