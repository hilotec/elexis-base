% *******************************************************************************
% * Copyright (c) 2007 by Elexis
% * All rights reserved. This document and the accompanying materials
% * are made available under the terms of the Eclipse Public License v1.0
% * which accompanies this distribution, and is available at
% * http://www.eclipse.org/legal/epl-v10.html
% *
% *  $Id: voraussetzungen.tex 3094 2007-09-04 10:03:03Z rgw_ch $
%
%*******************************************************************************
% !Mode:: "TeX:UTF-8" (encoding info for WinEdt)

\section{Mindestanforderungen an die Hardware}
\label{systemvoraussetzungen}
\begin{itemize}
 \item Ein halbwegs aktueller PC (mind. 1 Ghz Taktfrequenz, mind. 512 MB RAM
 (empfohlen: 1 GB), mind. 1GB Harddiskplatz
\item  Eine Grafikkarte, die mindestens 1024x768 Pixel (empfohlen: 1280x1024)
anzeigen kann und ein dazu passender Monitor (z.B. 17-Zoll TFT oder grösser).
\item Empfohlen: Ein Drucker mit Schacht für A5-Papier (Rezepte, AUF-Zeugnisse) und ein oder zwei A4-Schächten (Für Briefe und Rechnungen)
\item Empfohlen: Ein Etikettendrucker.
\item Empfohlen: Ein externes Laufwerk zur Datensicherung.
\item Empfohlen: Von einer \textbf{Hardware}-Firewall geschützter Internet-Zugang (Personal Firewall ist ausdrücklich \textbf{nicht} empfohlen) (s. S. \pageref{sicherheit}).
\end{itemize}

\section{Unterstützte Betriebssysteme}
Elexis ist prinzipiell unter jedem Betriebssystem lauffähig, für das ein Java Runtime Environment Version 1.5
oder höher erhältlich ist. Im Speziellen sind das:
\begin{itemize}
\item  Windows 2000, XP, Vista
\item  Macintosh OS ab 10.4 (Tiger)  \footnote{Unter MacOS-X ist leider keine
OpenOffice-Einbindung möglich}
\item Linux (SuSE ab 9.3 oder Xubuntu/Kubuntu ab 6.06)\footnote{Achtung: Bei Linux mit Gnome-Desktop (wie Ubuntu) funktioniert die OpenOffice-Einbindung nicht, KDE (Kubuntu) und Xfce (Xubuntu) sind ok.}
\end{itemize}
Auf diesen Betriebssystemen sollte sich Elexis direkt mit einer der bereitgestellten Komplettversionen installieren lassen. Für andere Betriebssysteme kann mehr oder weniger Handarbeit notwendig sein.

Bitte beachten Sie, dass unsere Pauschalangebote und Wartungsverträge jeweils nur mit einem System möglich sind, das obengenannte Hard- und  Softwareanforderungen erfüllt. 