% *******************************************************************************
% * Copyright (c) 2007 by Elexis
% * All rights reserved. This document and the accompanying materials
% * are made available under the terms of the Eclipse Public License v1.0
% * which accompanies this distribution, and is available at
% * http://www.eclipse.org/legal/epl-v10.html
% *
% * Contributors:
% *    G. Weirich
% *
% *  $Id: elexis.tex 2453 2007-05-30 15:16:13Z rgw_ch $
% *******************************************************************************
% !Mode:: "TeX:UTF-8" (encoding info for WinEdt)

\section{Standard-Installation}
\label{easyistall}
\index{installation}
Um den raschen Einstieg zu erleichtern haben wir für Windows, Linux und Mac leicht installierbare und sofort lauffähige Pakete zusammengestellt, die auch schon eine Beispiel-Datenbank enthalten.
Vergewissern Sie sich bitte zunächst, ob Ihr System von Elexis unterstützt wird (Anhang \ref{systemvoraussetzungen} dieses Handbuchs, Seite \pageref{systemvoraussetzungen})
\subsection{Windows 2000/XP}
(NB: Vista gehört noch nicht zu den unterstützten Betriebssystemen)
\begin{itemize}
	\item Laden Sie \href{http://www.elexis.ch/download.php?file=elexis-windows}{www.elexis.ch/down\-load.php?file=elexis-windows} herunter (ca. 150MB)
	\item starten Sie die heruntergeladene Datei elexis-windows-x.y.z.exe und folgen Sie den Anweisungen am Bildschirm.
    \item Programm starten: Doppelklick auf das Symbol \glqq Elexis\grqq{} auf Ihrem Desktop oder das Programm \glqq Elexis\grqq{} im Startmenü auswählen.
	\item Deinstallation: Klicken Sie auf \glqq Elexis deinstallieren\grqq im  Startmenü.
\end{itemize}

\subsection{Linux}
(getestet mit SuSE 10.0, Kubuntu 6.06, 6.10 und 7.04, sowie  Xubuntu 6.10 und
7.04.)
\begin{itemize}
	\item Laden Sie \href{http://www.rgw.ch/download.php?file=elexis-linux}{www.rgw.ch/download.php?file=elexis-linux} herunter (ca. 90MB)
	\item Entpacken Sie das	heruntergeladene Archiv	elexis-linux-x.y.z.tgz an eine beliebige Stelle\footnote{Z.B. mit dem Befehl tar -xzf elexis-linux-x.y.z.tgz}
    \item Programm starten: Starten Sie einfach das Programm \footnote{meistens so: ./elexis}. Selbstverständlich können Sie auch unter Linux einen Shortcut auf dem Desktop
    anlegen, das dazu nötige Vorgehen variiert aber je nach desktop environment.

 	\item Textprogramm einrichten: Da bei den empfohlenen Linux-Distributionen
 	OpenOffice ohnehin integriert ist, liefern wir es beim Linux-Installer nicht
 	mit. Daher ist dieser auch erheblich kleiner, als der Windows-Installer. Dafür
 	ist noch ein wenig 	\glqq Handarbeit\grqq{} notwendig, um die Zusammenarbeit
 	von Elexis mit OpenOffice einzurichten (Anleitung für Kubuntu/Xubuntu, SuSE
 	analog mit Yast statt mit apt-get):
	\begin{itemize}
	 	\item öffnen Sie eine Konsole und geben Sie ein: \textit{sudo apt-get in\-stall
	 	openoffice.org-office\-bean}
		\item öffnen Sie in Elexis das Menü \textsc{Datei - Einstellungen} und suchen
		Sie dort die Seite \textsc{Textverarbeitung} auf. Markieren Sie den Punkt
		\glqq Open\-Of\-fice Wrap\-per\grqq{}.
		\item Gehen Sie dann im selben Dialog zur Seite \textsc{OpenOffice.org}.
		Suchen Sie mit dem Knopf \textsc{Durchsuchen} Ihr OpenOffice-
		Programmverzeichnis auf (in aller Regel wird dies bei Kubuntu
		/usr/lib/openoffice/program sein). Klicken Sie dann auf \textsc{Anwenden} und
		schliessen Sie den Dialog mit \textsc{OK}
		\item Wichtig: Verlassen Sie Elexis, warten Sie einige Sekunden und starten
		Sie neu.
    \end{itemize}
  \item Deinstallation: Löschen Sie den beim Entpacken erstellten Ordner. Das
 ist alles.
\end{itemize}

\subsection{Apple Macintosh OS-X}\
(Getestet Version 10.4 (Tiger). Einschränkung: Kein integriertes Textsystem)
\begin{itemize}
	\item Laden Sie \href{http://www.elexis.ch/download.php?file=elexis-macosx}{www.elexis.ch/down\-load.php?file=elexis-macosx} herunter (ca. 25 MB)
	\item Entpacken Sie das heruntergeladene Archiv elexis-macosx-x.y.z.zip an eine beliebige Stelle.
    \item Programm starten: Öffnen Sie das herunter\-geladene Verzeichnis und doppel\-klic\-ken Sie auf das Pro\-gramm \glqq Elexis\grqq{}
	\item Deinstallation: Löschen Sie den beim Entpacken erstellten Ordner. Das ist alles.
\end{itemize}

Bei allen Betriebssystemen wird beim ersten Start die Einrichtung des Programms komplettiert.
Geben Sie als Anwender und als Passwort jeweils \textbf{\textit{test}}\index{Passwort}\index{Anfangspasswort}
ein und beenden Sie das Programm gleich wieder. Warten Sie nach dem Programmende mindestens 30 Sekunden (Aufbau der Demo-Datenbank). Beim nächsten Start sollte alles normal funktionieren (ebenfalls wieder als \glqq test\grqq mit dem Passwort \glqq test\grqq anmelden).

Wir empfehlen, dass Sie dann zum \glqq warmwerden\grqq{} die geführte Tour (s.S.
\pageref{tour}) machen.

\bigskip
Nach der Installation haben Sie ein System mit einer vorkonfigurierten einfachen
lokalen Demo-Datenbank und einigen Beispiel-Patienten. Wenn Sie dieses System in ein
\glqq echtes\grqq{} Praxisprogramm umwandeln wollen, lesen Sie bitte in Anhang
\ref{vollversion} (S. \pageref{vollversion}), wie das geht.
