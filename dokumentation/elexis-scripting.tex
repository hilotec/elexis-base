% *******************************************************************************
% * Copyright (c) 2008 by Elexis
% * All rights reserved. This document and the accompanying materials
% * are made available under the terms of the Eclipse Public License v1.0
% * which accompanies this distribution, and is available at
% * http://www.eclipse.org/legal/epl-v10.html
% *
% * Contributors:
% *    G. Weirich
% *
% *  $Id: elexis-privatrechnung.tex 3739 2008-03-23 20:20:40Z rgw_ch $
% *******************************************************************************
% !Mode:: "TeX:UTF-8" (encoding info for WinEdt)

\documentclass[a4paper]{scrartcl}
\usepackage{german}
\usepackage[utf8]{inputenc}
\usepackage{makeidx}
\makeindex
% Hier ein etwas skurriler Block, der dazu dient, die Unterschiede
% zwischen pdflatex und latex auszubügeln
% Grafiken müssen als png oder gif (für pdflatex) und als eps (für Latex)
% vorhanden sein. Die Endung kann man beim \includegraphics jeweils weglassen,
% das System nimmt je nach Renderer die geeignete Variante.

\newif\ifpdf
\ifx\pdfoutput\undefined
	\pdffalse              	%normales LaTeX wird ausgeführt
\else
	\pdfoutput=1
	\pdftrue               	%pdfLaTeX wird ausgeführt
\fi

\ifpdf
	\usepackage[pdftex]{graphicx}
	\DeclareGraphicsExtensions{.pdf,.jpg,.png}
\else
	\usepackage[dvips]{graphicx}
	\DeclareGraphicsExtensions{.eps}
\fi

\usepackage{floatflt}
\usepackage{wrapfig}
\usepackage[]{hyperref}
\usepackage{color}
\begin{document}
\title{Elexis-Scripting}
\author{Gerry Weirich}
\maketitle

\section{Einführung}
Scripting ist eine sehr mächtige Möglichkeit, die Funktionalität von Elexis selbst zu erweitern, ohne in den Programmcode eingreifen zu müssen. Ein Script ist eine Art Miniprogramm, ds innerhalb von Elexis ausgeführt wird, und das Zugriff auf alle von Elexis verwalteten Daten hat.

\medskip

Scripts können an verschiedenen Stellen eingesetzt werden. Beispielsweise bei der Berechnung von formelbasierten Laborwerten oder Befunden, oder bei der Filterung der Patientenliste, oder auch als Selbständige Funktionen.

\section{Sprache und Syntax}
Die zu verwendende Sprache ist ein vereinfachtes Java (aber auch 'richtiges' Java geht). Im Genensatz zu einem 'echten' Java-Programm brauchen Sie keinen Compiler. Stattdessen wird das Script Interpretiert, also in dem Moment analysiert und in Computerbefehle umgesetzt, in dem es ausgeführt wird. Ein genauere Erläuterung des Interpreters finden Sie bei dessen Hersteller Beanshell (http://www.beanshell.org). Hier nur eine ganz kurze Einführung:

\subsection{Grundsätzlicher Aufbau}

\begin{verbatim}
    /* Dies ist ein Beispiel-Script. Text, welcher zwischen diesen Kommentarsymbolen
       steht, wird als Kommentar betrachtet, dient also nur der Erhellung oder
       Erheiterung des menschlichen Lesers. Der Interpreter hingegen ignoriert
       solchen Text schlicht. Sie sollten bei komplexeren Scripts nicht mit
       Kommentaren sparen, da sie Ihnen helfen, später zu verstehen, was Sie
       eigentlich machen wollten. Und anderen kann es helfen, Ihr Script zu
       verstehen.
    */

    // Auch dies ist ein Kommentar, aber bei // geht der Kommentar immer nur
    // bis zum Ende der aktuellen Zeile, bei /* immer bis zum nächsten */

    irgendwas=10;
        /* Wir haben soeben eine Variable namens 'irgendwas' erstellt,
           und ihr den Wert 10 zugewiesen. Beachten Sie bitte auch, dass
           am Ende jedes Statements ein ; steht, damit der Computer weiss,
           dass das Statement hier fertig ist. */

    heute="Mittwoch";
        /* Hier haben wir eine Variable namens 'heute' erstellt und ihr
           den Wert 'Mittwoch' zugewiesen. Beachten Sie, dass Zeichenketten
           immer in Anführungszeichen stehen müssen. */

    grosseZahl=1000;    // Ich denke, das ist klar.

    zusammen=irgendwas+grosseZahl;
        // 'zusammen' sollte jetzt 1010 sein, wenn Ihr Computer nicht kaputt ist

        // wir wollen das gleich mal testen:
    if(zusammen==1010){
                // Geschweifte Klammern markieren zusammengehörende Blöcke
        resultat="Computer rechnet richtig";
        wertDesComputers=1000;
    }else{
        resultat="Computer rechnet falsch";
        wertDesComputers=0;
    }

        // Ihr Computer kann natürlich auch multiplizieren und dividieren
    produkt=irgendwas*grosseZahl*3.4;
    quotient=grosseZahl/(irgendwas-2);

        // Einfaches Rechnen mit Variablen
    irgendwas=irgendwas+5;     // irgendwas ist jetzt 15
    irgendwas+=3;               // irgendwas ist jetzt 18
    irgendwas++;                // irgendwas ist jetzt 19

    heute=heute+" 18. Juni";    // 'heute' ist jetzt "Mittwoch 18. Juni"


\end{verbatim}

\section{Verbindung mit Elexis}
Ein Script kann Verbindung mit Elexis aufnehmen, indem es in Elexis definierte Java-Objekte
verwendet. Um dies auszunützen, braucht man allerdings gewisse Kennntnisse von Elexis-Interna.

Es ist daher wohl sinnvoll zunächst einfach die in Beispielen gezeigten Objekte zu verwenden.
Auf diese Weise stellt sich das nötige Kennenlernen der sinnvoll verwendbaren Objekte mit der
Zeit von allein ein. Oder man lässt sich benötigte Scripts von jemand anderem erstellen.

\medskip

Ach so, wenn Sie mit dem Begriff 'Objekt' nichts anfangen können, dann nur soviel: Ein Objekt ist (bei objektorientierten Programmiersprachen wie Java) ein Programm-Konstrukt, das bestimmte Eigenschaften und Fähigkeiten (sog. 'Methoden') hat. Beispiel:
\begin{small}
\begin{verbatim}
 import de.volkswagen.autos.*;  // importiere die benötigten Klassen

 meinAuto=new Kaefer();         // erstelle ein Objekt einer bestimmten Klasse
 meinAuto.setFarbe("Grün");     // setze Eigenschaften des Objekts
 if(meinAuto.istKaputt()){      // frage eine Eigenschaft des Objekts ab
     // wir bleiben zuhause
 }else{
     meinAuto.fahreNach("Bern"); // führe eine Methode des Objekts aus.
 }

\end{verbatim}
\end{small}

Beachten Sie in obigem Beispiel, dass uns die genaue Implementation des 'Kaefers' nicht zu interessieren braucht. Wir müssen weder wissen, wie er hergestellt wird, noch woraus er besteht. Wir müssen nur wissen, welche seiner Eigenschaften und Methoden für uns interessant sind.
Das macht Objekte so nützlich in der Programmierung. Man muss nicht alles selber machen, sondern
kann die Vorarbeiten anderer (oder eigene frühere Vorarbeiten) nutzen.

\end{document} 