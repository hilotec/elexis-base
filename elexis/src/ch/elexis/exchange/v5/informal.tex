%%%%%%%%%%%%%%%%%%%%%%%%%%%%%%%%%%%%%%%%%%%%%%%%%%%%%%%%%%%%%%%%%%%%%%%%%%%%%%%%%%%%%%%%%%%%%%
% This is an informal description of the xChange 5.0x standard for sharing medical documents
% The formal description can be found in xChange_501.xsd
% a sample file using this schema is provides as sample.xml
% $Id$
%%%%%%%%%%%%%%%%%%%%%%%%%%%%%%%%%%%%%%%%%%%%%%%%%%%%%%%%%%%%%%%%%%%%%%%%%%%%%%%%%%%%%%%%%%%%%%
% !Mode:: "TeX:UTF-8" (encoding info for WinEdt)

\documentclass[a4paper]{scrartcl}
\usepackage{german}
\usepackage[utf8]{inputenc}
\usepackage{makeidx}
\makeindex

\usepackage{floatflt}
\usepackage[]{hyperref}
\usepackage{color}
\begin{document}
\title{xChange - Dokumentation}
\maketitle
\section{Einleitung}
Dies ist ein vorläufiger Entwurf des xChange 5.0x Standards zum Austausch medizinischer Daten
Dieser Entwurf richtet sich an:
\begin{itemize}
    \item Behörden und Institutionen, welche ein einheitliches Datenformat fördern und unterstützen möchten.
    \item Entwickler von im Medizinbereich verwendeter Software, welche die Möglichkeit einsetzen möchten, Daten mit anderer Software auszutauschen.
\end{itemize}

\subsection{Umfang der Dokumentation}
Die folgenden Dateien sind integraler Bestandteil dieser Dokumentation und des Standards:
\begin{itemize}
    \item xChange\_501.xsd: XML-Schema, das den Standard beschreibt
    \item xChange.xml: Eine Beispieldatei, welche den Standard benutzt
\end{itemize}

\subsection{Copyright}
xChange 5.0x ist \copyright 2007 by Forum Datenaustausch, Schweiz. Die Verwendung des Standards zum Zweck des Medizinischen Datenaustauschs ist für jedermann frei.
Jede Änderung oder Erweiterung des Standards bedarf der schriftlichen Einwilligung des Copyright-Inhabers. 

\section{Designziele}
xChange soll den Datenaustausch zwischen medizinischen Anwendungsprogrammen ermöglichen. Es soll hierbei möglich sein, mit ein- und demselben Container und Datenformat ganze Krankengeschichten (KG's), oder auch beliebige Subsets hiervon zu transportieren. Ebenso soll es möglich sein, nicht-patientenbezogene Daten zu transportieren. Wesentliche Punkte sind:

\begin{itemize}
  \item Soweit möglich Übertragung der vollständigen Informationen
  \item Einfach (und damit preiswert) implementierbar
  \item Einfach an existierende elKG-Software anzupassen
  \item Keine Abhängigkeiten von proprietärer Software oder nicht-freier Standards. Dies ist nicht immer völlig konsequent durchführbar, da häufig der Wunsch besteht, Dokumente im Originalformat zu transferieren (z.B. Microsoft\texttrademark Word\texttrademark Dateien).
  \item Keine Belastung durch Lizenzgebühren oder Nutzungsbeschränkungen
  \item Komplett oder partiell verschlüsselbar
\end{itemize}

\section{Aufbau}
\subsection{Container}
Eine xChange-Datei ist eine gepackte Datei im ZIP-Format\footnote{Referenz noch zu nennen}, welche folgende Elemente enthält:
\begin{itemize}
\item \textit{Genau eine} XML-Datei (das Zentraldokument), welche Metadaten und reine Text-Daten des Austauschs enthält
\item Keine oder beliebig viele binäre Dateien, auf die im Zentraldokument verwiesen wird.
\end{itemize}
Die Binärdateien enthalten z.B. Bilder oder andere Dokumente in Binärformaten (z.B. PDF), welche sich nicht effizient in Textform abbilden lassen.

subsection{Zentraldokument}

\end{document}